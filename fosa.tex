\documentclass[10pt]{article}
\usepackage[english, ngerman]{babel}
\usepackage[utf8]{inputenc}
\usepackage{xcolor}
\usepackage[a4paper,margin=2.5cm]{geometry} \usepackage{blindtext}
\usepackage{setspace}
\usepackage{float}
\usepackage{titletoc}
\usepackage{titlesec}
\usepackage{wrapfig}
\usepackage{amsmath}
\usepackage{multicol}

\onehalfspacing
\usepackage[hidelinks]{hyperref}

\titleformat{\section}
  {\normalfont\fontsize{16}{15}\bfseries}{\thesection}{1em}{}

\titleformat{\subsection}
  {\normalfont\fontsize{14}{15}\bfseries}{\thesubsection}{1em}{}

\titleformat{\subsubsection}
  {\normalfont\fontsize{12}{15}\bfseries}{\thesubsubsection}{1em}{}


\begin{document}
\title{Statistik Formelsammlung}
\author{Katharina Ring}
\date{\today}
\maketitle

\clearpage

\tableofcontents

\clearpage

%-------------------------------------------------------------------------------

% SECTION: DESKRIPTIVE STATISTIK

%-------------------------------------------------------------------------------

\section{Deskriptive Statistik}

\subsection{Kenngrößen (Parameter)}
\subsubsection{Lagemaße}
\begin{multicols}{2}

\paragraph{Arithmetisches Mittel}

 $$\bar{x}=\frac{1}{n}\sum\limits_{i=1}^n x_i$$

\noindent Schätzer für den Erwartungswert 
$\mu = E[X]$


\paragraph{Stichprobe}

    Arithmetisches Mittel
    Geometrisches Mittel
    Harmonisches Mittel
    Quantile: Median, Quartile, Dezile, Perzentile
    Modus
\end{multicols}
\subsubsection{Streuungsmaße}



    Stichprobenvarianz $s$
    Standardabweichung $\sigma$
    Spannweite R (engl. range)
    Mittlere absolute Abweichung X
    (Inter-)Quartilsabstand Q (engl. interquartile range; Abk. IQR)

\subsubsection{Konzentrationsmaße}


    Absolute Konzentration
    Relative Konzentration
    Atkinson-Maß
    Gini-Koeffizient aus der Lorenz-Kurve
    Herfindahl-Index (Herfindahl)
    Hoover-Ungleichverteilung
    Rosenbluth-Index
    Theil-Index


\subsubsection{Gestaltmaße}


    Schiefe (Statistik)
    Wölbung (Statistik)


\subsubsection{Zusammenhangsmaße}


    2.1 Für zwei nominale Variablen
    2.2 Für zwei ordinale Variablen
    2.3 Für zwei metrische Variablen
    2.4 Für zwei Variablen unterschiedlichen Skalenniveaus

(Wiki)

\subsection{Tabellen}

\subsection{Diagramme}


%-------------------------------------------------------------------------------

% SECTION: WAHRSCHEINLICHKEITSRECHNUNG

%-------------------------------------------------------------------------------

\section{Wahrscheinlichkeitsrechnung}


%-------------------------------------------------------------------------------

% SECTION: HYPOTHESENTESTS

%-------------------------------------------------------------------------------

\section{Hypothesentests}

%-------------------------------------------------------------------------------

% SECTION: REGRESSION

%-------------------------------------------------------------------------------

\section{Regression}

\subsection{Annahmen}

\subsection{Verfahren}
\subsubsection{Kleinste Quadrate (OLS)}
\subsubsection{Maximum Likelihood}

\subsection{c}

\subsection{d}

%-------------------------------------------------------------------------------

% SECTION: KLASSIFIKATION

%-------------------------------------------------------------------------------

\section{Klassifikation}

\subsection{Diskriminanzanalyse (Bayes)}

\subsection{Clusteranalyse}


\end{document}
