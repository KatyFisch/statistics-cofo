% To Do: no pagebreak (in multicol) for paragraphs
%        Find out Formula for Kurtosis
%        raggedbottom funktioniert nicht
%        commentare - two versions
\documentclass[8pt]{extarticle}
\usepackage[english, ngerman]{babel}
\usepackage[utf8]{inputenc}
\usepackage{xcolor}
\usepackage[a4paper,left=2.3cm,right=1.2cm,top=2cm,bottom=2cm]{geometry} 
\usepackage{blindtext}
\usepackage{setspace}
\usepackage{float}
\usepackage{titletoc}
\usepackage{titlesec}
\usepackage{wrapfig}
\usepackage{amsmath}
\usepackage{multicol}
\usepackage{amsfonts}
\onehalfspacing
\usepackage[hidelinks]{hyperref}
\usepackage[all]{nowidow} %funktioniert nicht....

% Inhaltsverzeichnis mit zwei Spalten
\usepackage[toc]{multitoc}
\renewcommand*{\multicolumntoc}{2}


%Überschriftengrößen anpassen, so dass Paragraph kleiner ist als Subsubsection
\titleformat{\section}
  {\normalfont\fontsize{16}{15}\bfseries}{\thesection}{1em}{}
\titleformat{\subsection}
  {\normalfont\fontsize{14}{15}\bfseries}{\thesubsection}{1em}{}
\titleformat{\subsubsection}
  {\normalfont\fontsize{12}{15}\bfseries}{\thesubsubsection}{1em}{}


\begin{document}
\title{\Huge Statistik Formelsammlung}
\author{\LARGE Katharina Ring}
\date{\LARGE \today}
\Huge \maketitle \normalsize

\clearpage

\tableofcontents

\clearpage


% weitere Anpassungen im Hauptteil des Dokuments
\raggedright %linksbündig
\setlength{\parindent}{15pt} %Einzuglänge festsetzen
\setlength{\columnseprule}{0.3pt} %Liniendicke zwischen zwei Multicols
\raggedbottom %funktioniert nicht




%-------------------------------------------------------------------------------

% SECTION: DESKRIPTIVE STATISTIK

%-------------------------------------------------------------------------------

\section{Deskriptive Statistik}



\subsection{Kenngrößen (Parameter)}

\begin{multicols}{2}[\subsubsection{Lagemaße}][4cm] 

\paragraph{Modus}

 Häufigster Wert von $x_i$. Auch zwei oder mehr Modi sind möglich (bimodal).

\paragraph{Median}

$$\tilde{x}_{0.5}=\begin{cases} x_{((n+1)/2)} & \text{falls n ungerade} \\ \frac{1}{2}(x_{(n/2)}+x_{(n/2+1)} & \text{falls n gerade} \end{cases}$$

\paragraph{Quantile}

$$\tilde{x}_\alpha=\begin{cases} x_{(k)} & \text{falls n}\alpha \notin \mathbb{N}\\ \frac{1}{2}(x_{(n\alpha)}+ x_{(n\alpha+1)}) & \text{falls n}\alpha \text{ ganzzahlig} \end{cases}$$

mit

\begin{tabular}{l l}
 k=min x $\in \mathbb{N}$, &  x $>$ n$\alpha$ \\
\end{tabular}

\paragraph{Minimum/Maximum}


$$x_{\min}=\min_{i \in \{ 1,...,N\}} (x_i) \hspace{0.8cm}   x_{\max}=\max_{i \in \{ 1,...,N\}} (x_i)$$
 


\paragraph{Arithmetisches Mittel}

 $$\bar{x}=\frac{1}{n}\sum\limits_{i=1}^n x_i$$

\noindent Schätzer für den Erwartungswert 
$\mu = E[X]$ (erstes~Verteilungsmoment)

\paragraph{Geometrisches Mittel}

$$\bar{x}_G=\sqrt[n]{\sum\limits_{i=1}^n x_i} $$

\noindent Für Wachstumsfaktoren: $\bar{x}_G=\sqrt[n]{\frac{B_n}{B_0}}$

\paragraph{Harmonisches Mittel}

$$\bar{x}_H=\frac{\sum\limits_{i=1}^n w_i}{\sum\limits_{i=1}^n \frac{w_i}{x_i}}$$


\end{multicols}


\begin{multicols}{2}[\subsubsection{Streuungsmaße}][4cm] 

\paragraph{Spannweite}

$$R=x_{(n)}-x_{(1)}$$

\paragraph{Quartilsabstand}

$$d_Q=\tilde{x}_{0.75}-\tilde{x}_{0.25}$$

\paragraph{(Empirische) Varianz}

$$s^2=\frac{1}{n}\sum\limits_{i=1}^n(x_i-\bar{x})^2=\frac{1}{n}\sum\limits_{i=1}^nx_i^2-\bar{x}^2$$

\noindent Schätzer für das zweite zentrierte Moment, inkl. Varianzverschiebungssatz

\paragraph{(Empirische) Standardabweichung}

$$s=\sqrt{s^2}$$

\paragraph{Variationskoeffizient}

$$ \nu=\frac{s}{\bar{x}}$$

\paragraph{Mittlere absolute Abweichung}


$$ \mathit{e} = \frac{1}{n}\sum_{i=1}^n \left|x_i - \bar{x}\right|$$

\noindent Schätzer für das erste absolute zentrierte Moment 

\end{multicols}



\begin{multicols}{2}[\subsubsection{Konzentrationsmaße}][4cm] 

\paragraph{Gini-Koeffizient}

\begin{equation*} 
\begin{split}
G & = \frac{2\sum\limits_{i=1}^n ix_{(i)}-(n+1)\sum\limits_{i=1}^n x_{(i)}}{n\sum\limits_{i=1}^n x_{(i)}}  = 1-\frac{1}{n}\sum\limits_{i=1}^n(v_{i-1}+v_i)
\end{split}
\end{equation*}

mit

$$  u_i=\frac{i}{n}, \hspace{0.3cm} v_i= \frac{\sum\limits_{j=1}^i x_{(j)}}{\sum\limits_{j=1}^i x_{(j)}} \hspace{0.7cm} (u_0=0, \hspace{0.2cm} v_0=0 )$$


\noindent Dies sind auch die Werte für die Lorenzkurve.

\ \\

\indent Wertebereich: $ 0 \le G \le \frac{n-1}{n}$




\paragraph{Lorenz-Münzner-Koeffizient ($G$ normiert)}

$$G^+=\frac{n}{n-1}G$$

\indent Wertebereich: $ 0 \le G^+ \le 1$






\end{multicols}




\begin{multicols}{2}[\subsubsection{Gestaltmaße}][4cm] 
\paragraph{(Empirische) Schiefe}

$$\nu = \frac{n}{(n-1)(n-2)} \sum_{i=1}^n \left(\frac{x_i-\bar{x}}{s}\right)^3$$

\noindent Schätzer für das dritte zentrierte Moment, normiert durch $(\sigma^2)^{\frac{2}{3}}$

\paragraph{(Empirische) Wölbung/Kurtosis}

$$k=\left[n(n+1) \cdot \sum_{i=1}^n \left(\frac{x_i-\bar{x}}{s}\right)^4 - 3(n-1)\right] \cdot \frac{n-1}{(n-2)(n-3)}+3$$

\noindent Schätzer für das vierte zentrierte Moment, normiert durch $(\sigma^2)^2$

\paragraph{Exzess}

$$\gamma=k-3$$

\end{multicols}



\begin{multicols}{2}[\subsubsection{Zusammenhangsmaße}][4cm]

\subsubsection*{\textit{Für zwei nominale Variablen}}

\paragraph{$\chi^2$-Statistik}

\begin{equation*}
\begin{split}
\chi^2 & =\sum\limits_{i=1}^k \sum\limits_{j=1}^l \frac{(n_{ij}-\frac{n_{i+}n_{+j}}{n})^2}{\frac{n_{i+}n_{+j}}{n}}  =n\left(\sum\limits_{i=1}^k \sum\limits_{j=1}^l \frac{n_{ij}^2}{n_{i+}n_{+j}}-1\right)
\end{split}
\end{equation*}

Wertebereich: $ 0 \le \chi^2 \le n(\min(k,l)-1)$

\paragraph{Phi-Koeffizient}

$$\Phi=\sqrt{\frac{\chi^2}{n}}$$

Wertebereich: $ 0 \le \Phi \le \sqrt{\min(k,l)-1}$

\paragraph{Cram\'ers $V$}

$$ V= \sqrt{\frac{\chi^2}{\min(k,l)-1}}$$

Wertebereich: $ 0 \le V \le 1$

\paragraph{Kontingenzkoeffizient $C$}

$$C=\sqrt{\frac{\chi^2}{\chi^2 + n}}$$

Wertebereich: $ 0 \le C \le \sqrt{\frac{\min(k,l)-1}{\min(k,l)}} $

\paragraph{Korrigierter Kontingenzkoeffizient $C_{korr}$}

$$C_{korr}= \sqrt{\frac{\min(k,l)}{\min(k,l)-1}} \cdot \sqrt{\frac{\chi^2}{\chi^2 + n}} $$

Wertebereich: $ 0 \le C_{korr} \le 1 $

\paragraph{Odds-Ratio}

$$OR=\frac{ad}{bc} = \frac{n_{ii}n_{jj}}{n_{ij}n_{ji}}$$

Wertebereich: $0 \le OR < \infty$

\subsubsection*{\textit{Für zwei ordinale Variablen}}

\paragraph{Gamma nach Goodman und Kruskal}

$$\gamma=\frac{K-D}{K+D}$$


\begin{tabular}{l l }
$ K=\sum_{i<m}\sum_{j<n}n_{ij}n_{mn}$ & Anzahl konkordanter Paare \\
$ D=\sum_{i<m}\sum_{j>n}n_{ij}n_{mn}$ & Anzahl diskordanter Paare \\
\end{tabular}

\ \\

Wertebereich: $-1 \le \gamma \le 1$

\paragraph{Kendalls $\tau_b$}

$$ \tau_b=\frac{K-D}{\sqrt{(K+D+T_X)(K+D+T_Y)}}$$

mit

\begin{tabular}{l l } 
$ T_X=\sum_{i=m}\sum_{j<n}n_{ij}n_{mn}$ & Anzahl Bindungen bzgl. $X$ \\
$ T_Y=\sum_{i<m}\sum_{j=n}n_{ij}n_{mn}$ & Anzahl Bindungen bzgl. $Y$ \\
\end{tabular}

\ \\

Wertebereich: $-1 \le \tau_b \le 1$

\paragraph{Kendalls/Stuarts $\tau_c$}

$$\tau_c=\frac{2\min(k,l)(K-D)}{n^2(\min(k,l)-1)}$$

Wertebereich: $-1 \le \tau_c \le 1$

\paragraph{Spearmans Rangkorrelationskoeffizient}

$$R=\frac{n(n^2-1)-\frac{1}{2}\sum\limits_{j=1}^J b_j(b_j^2-1)-\frac{1}{2}\sum\limits_{k=1}^K c_k(c_k^2-1)-6\sum\limits_{i=1}^n d_i^2}{\sqrt{n(n^2-1)-\sum\limits_{j=1}^J b_j(b_j^2-1)}\sqrt{n(n^2-1)-\sum\limits_{k=1}^Kc_k(c_k^2-1)}}$$

 Entspricht ohne Bindungen:

$$R=1-\frac{6\sum\limits_{i=1}^nd_i^2}{n(n^2-1)}$$

mit

\begin{tabular}{l l } 
 $d_i=R(x_i)-R(y_i)$ & Rangdifferenz \\ 
\end{tabular}

\ \\

Wertebereich: $-1 \le R \le 1$

\subsubsection*{\textit{Für zwei metrische Variablen}}

\paragraph{Korrelationskoeffizient nach Bravais-Pearson}

$$r=\frac{S_{xy}}{\sqrt{S_{xx}S_{yy}}}=\frac{s_{xy}}{\sqrt{s_{xx}s_{yy}}}$$

mit

\begin{tabular}{l l } 
$S_{xy}=\sum\limits_{i=1}^n(x_i-\bar{x})^2(y_i-\bar{y})^2$ & bzw. $s_{xy}=\frac{S_{xy}}{n}$ \\
$S_{xx}=\sum\limits_{i=1}^n(x_i-\bar{x})^2$ & bzw. $s_{xx}=\frac{S_{xx}}{n}$ \\ 
$S_{yy}=\sum\limits_{i=1}^n(y_i-\bar{y})^2$ & bzw. $s_{yy}=\frac{S_{yy}}{n}$ \\
\end{tabular}

\ \\

Wertebereich: $-1 \le r \le 1$


%\subsubsection*{\textit{Für zwei unterschiedliche Variablen}}


\end{multicols}

\subsection{Tabellen}

\subsection{Diagramme}


%-------------------------------------------------------------------------------

% SECTION: WAHRSCHEINLICHKEITSRECHNUNG

%-------------------------------------------------------------------------------

\section{Wahrscheinlichkeitsrechnung}


%-------------------------------------------------------------------------------

% SECTION: HYPOTHESENTESTS

%-------------------------------------------------------------------------------

\section{Hypothesentests}

%-------------------------------------------------------------------------------

% SECTION: REGRESSION

%-------------------------------------------------------------------------------

\section{Regression}

\subsection{Annahmen}

\subsection{Verfahren}
\subsubsection{Kleinste Quadrate (OLS)}
\subsubsection{Maximum Likelihood}

\subsection{c}

\subsection{d}

%-------------------------------------------------------------------------------

% SECTION: KLASSIFIKATION

%-------------------------------------------------------------------------------

\section{Klassifikation}

\subsection{Diskriminanzanalyse (Bayes)}

\section{Clusteranalyse}


\end{document}
